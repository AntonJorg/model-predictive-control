\chapter{Control Structure}

Consider the Modified Four Tank System. Assume that we measure the levels in tank 1, tank 2. The flows F1 and F2 can be controlled by manipulating the two pumps. Assume that the flows F3 and F4 are unmeasured flow rates that we cannot manipulate. F3 and F4 are stochastic variables (normally distributed). We want to control the levels in tank 1 and tank 2 to desired set points ¯z1 and ¯z2.

\section{What are the states, x, the measurement, y, the manipulated variables (MVs), u, the measured disturbance variables (DVs), and the controlled variables (CVs) for this system?}

\section{Draw a block-diagram of the Modified Four Tank System with an MPC system. The MPC block should illustrate both the state-estimator and regulator of the MPC.}


\iffalse
Example on how to cite
\cite{Nocedal:Wright:2006}

\section{Section XX}
Example on how to write an optimization problem
\begin{subequations}
\begin{alignat}{3}
& \min_x \quad && f(x) = \frac{1}{2} x' H x + g' x \\
& s.t. && c(x) = A' x + b = 0
\end{alignat}
\end{subequations}

\subsection{Subsection XX.aa}

\section{Section YY}

\begin{figure}[!b]
    \centering
    \includegraphics[width=0.75\textwidth, keepaspectratio=true]{figures/FsTsEaR.pdf}
\caption{Steady state temperature, $T_s$, as a function of the steady state flow rate, $F_s$ with varying $E_a/R$.}\label{fig:FsTsEaR}
\end{figure}


\begin{figure}[!tb]
\centering
    \begin{subfigure}[b]{0.3\textwidth}
         \centering
         \includegraphics[width=\textwidth]{figures/FsTsconst_flow450.pdf}
         \caption{Steady state temperature, $T_s$, as a function of the steady state flow rate, $F_s$.}
         \label{fig:FsTsconst_flow450}
     \end{subfigure}
     \hfill
     \begin{subfigure}[b]{0.3\textwidth}
         \centering
         \includegraphics[width=\textwidth]{figures/RTQ-T.pdf}
         \caption{The energy produces, $R_T(T)$, and the energy released, $Q(T)$, as functions of the temperature, $T$.}
         \label{fig:RTQ-T}
     \end{subfigure}
     \hfill
     \begin{subfigure}[b]{0.3\textwidth}
         \centering
         \includegraphics[width=\textwidth]{figures/dTdt-T.pdf}
         \caption{The change in temperature with respect to time, $\dot T$, as a function of the temperature, $T$.}
         \label{fig:dTdt-T}
     \end{subfigure}
        \caption{The dynamical behaviour of the system at flow rate, $F=450$ mL/min. MAKE SURE YOUR FIGURE TEXT IS LARGE ENOUGH TO BE READABLE.}
        \label{fig:flow rate 450}
\end{figure}


\section{Algorithm}

\begin{algorithm}
\caption{The explicit Euler algorithm}\label{alg:explicit Euler}
\KwData{The differential equation $f(t,x(t),p)$, initial and final time $t_0$ and $t_f$, number of steps $N$, initial conditions $x_0$, parameters $p$}
\KwResult{time and approximation vectors $T$ and $X$}
calculate step size, $dt=\frac{t_f-t_0}{N}$;\\

\For{n = 1 to N}{
$f=f(t_n,x_n,p)$\tcp*[r]{Evaluate $f$ at current step}
$t_{n+1}=t_n+dt$\tcp*[r]{Update and store time step}
$x_{n+1}=x_n+f\cdot dt$\tcp*[r]{Compute and store the next value of $x$}}
\end{algorithm}

\section{Matlab code}

\begin{lstlisting}[caption=Matlab code example]
clear
close all
clc

% Explicit Euler fixed step size for 1D CSTR
T = 0;
X = x0;
x = x0;
for i = 1:length(t)-1
    [Tn, Xn] = ExplicitEulerFixed(@CSTR1dDrift, [t(i) t(i+1)],N, x(i), U(i), p);
    x(i+1) = Xn(end);
    T = [T;Tn(2:end)];
    X = [X;Xn(2:end)];
end
\end{lstlisting}
\label{lst:MatlabCodeLabel1}

\fi