\chapter{Control Structure}

We consider the Modified Four Tank System~[\cite{johansson2002quadruple}] where the liquid levels in tanks 1 and 2 are measured, the pump flows $F_1$ and $F_2$ are manipulated, and the inflows $F_3$ and $F_4$ enter as unmeasured stochastic disturbances. The overall control objective is to regulate the lower-tank levels to the setpoints $\bar z_1$ and $\bar z_2$.

\section{1.1 States, Measurements, MVs, DVs, and CVs}
A standard choice is to define the states as the stored liquid amounts in each tank (equivalently, the levels). Using the lecture notation, one convenient state vector is the tank masses
\begin{align}
x(t) &= \begin{bmatrix} m_1(t) & m_2(t) & m_3(t) & m_4(t) \end{bmatrix}^\top,
\end{align}
where $m_i$ is the liquid mass in tank $i$. This choice is directly aligned with the mass-balance model formulation for the modified four-tank system.

\paragraph{Measurements.}
We measure the levels in tanks 1 and 2, so the measurement vector is
\begin{align}
y(t) &= \begin{bmatrix} h_1(t) & h_2(t) \end{bmatrix}^\top
= \begin{bmatrix} \dfrac{m_1(t)}{\rho A_1} & \dfrac{m_2(t)}{\rho A_2} \end{bmatrix}^\top,
\end{align}
where $A_i$ is the tank cross-sectional area and $\rho$ is the density of the liquid. (In a stochastic model one may add measurement noise $v(t)$, but the variable classification is unchanged.)

\paragraph{Manipulated variables (MVs).}
The manipulated variables are the pump flows that we can command:
\begin{align}
u(t) &= \begin{bmatrix} F_1(t) & F_2(t) \end{bmatrix}^\top.
\end{align}
These are the two degrees of freedom available to the controller.

\paragraph{Disturbance variables (DVs).}
The problem states that $F_3$ and $F_4$ are \emph{unmeasured} stochastic flow rates that we cannot manipulate. Thus, they are disturbances:
\begin{align}
d(t) &= \begin{bmatrix} F_3(t) & F_4(t) \end{bmatrix}^\top,
\end{align}
with $F_3(t),F_4(t)$ modeled as random variables/processes (e.g., piecewise constant random signals or continuous-time stochastic processes, depending on the chosen disturbance model).

\paragraph{Controlled variables (CVs).}
The controlled variables are the quantities we want to regulate to setpoints, i.e., the lower-tank levels in tanks 1 and 2:
\begin{align}
z(t) &= \begin{bmatrix} z_1(t) & z_2(t) \end{bmatrix}^\top
= \begin{bmatrix} h_1(t) & h_2(t) \end{bmatrix}^\top,
\end{align}
with corresponding setpoints
\begin{align}
\bar z &= \begin{bmatrix} \bar z_1 & \bar z_2 \end{bmatrix}^\top .
\end{align}
In this setup, the CVs coincide with a subset of the measurements, since we directly measure the variables we aim to control.

\section{Block Diagram of the Plant with MPC (Estimator + Regulator)}

The closed-loop structure consists of the plant (Modified Four Tank System) driven by manipulated inputs $u$ and unmeasured disturbances $d$, a sensor block producing measurements $y$, and an MPC block that contains (i) a state estimator and (ii) a regulator (optimizer).

\begin{figure}[!h]
\centering
\begin{tikzpicture}[
  block/.style={draw, rounded corners, minimum height=10mm, minimum width=22mm, align=center},
  big/.style={draw, rounded corners, inner sep=8mm},
  plant/.style={draw, minimum height=20mm, minimum width=30mm, align=center},
  line/.style={->, thick},
  line2/.style={-, thick}
]

% --- MPC inner blocks ---
\node[block] (ocp) {Linear\\regulator\\(OCP)};
\node[block, below=10mm of ocp] (est) {State\\estimator\\(KF)};

% --- MPC outer frame ---
\node[big, fit=(ocp)(est), label=above:{MPC}] (mpc) {};

% --- Plant block ---
\node[plant, right=40mm of ocp] (plant) {Modified\\4-Tank Plant};

% --- Plant block ---
\node[block, below=10mm of plant] (sensor) {Sensor model};

% --- Signals inside MPC ---
\draw[line2] (ocp.west) ++(-12mm,0) -- (ocp.west);
\node[left=16mm of ocp.west] (zref) {$\bar z$};
\draw[line, thick] (zref) -- (ocp.west);

% Estimator -> OCP (xhat)
\draw[line, thick] (est.north) -- node[right] {$\hat x$} (ocp.south);

% --- MPC -> Plant (u) ---
\draw[line, thick] (ocp.east) -- node[above] {$u$} (plant.west);
\draw[line, thick] (ocp.east) ++(4mm, 0) |- (est.east);

% --- Disturbance into plant (d) ---
\draw[line, thick] (plant.north) ++(0,5mm) node[above] {$d$} -- (plant.north);

% --- Plant output z to the right ---
\draw[line, thick] (plant.east) -- ++(10mm,0) node[right] {$z$};

% --- Plant output x to sensor ---
\draw[line, thick] (plant.south) -- node[right] {$x$} (sensor.north);

% --- Measurement y feedback (plant -> estimator) ---
% take a tap from plant right side (lower-ish) and route back under to estimator
\coordinate (ytap) at (sensor.south);
\draw[line, thick] (ytap) -- node[right] {$y$} ($(sensor.south)+(0,-12mm)$) -| (est.south);

\end{tikzpicture}
\caption{Block diagram of the MPC-controlled Modified Four Tank System.}\label{fig:system_diagram}
\end{figure}

For an in-depth description of the system, as well as MPC of a physical realization hereof, see~\cite{schroll2017implementation}.