\chapter{Deterministic and Stochastic Nonlinear Modeling}

\section{Deterministic Nonlinear Model}

\subsection{State definition}

We model each tank as perfectly mixed with constant cross-sectional area $A_i$ and liquid density $\rho$.
Let $m_i(t)$ denote the liquid mass in tank $i$ and $h_i(t)$ the corresponding liquid level.
Since the tank volume is $V_i = A_i h_i$ and $m_i = \rho V_i$, we have
\begin{equation}
    m_i(t) = \rho A_i h_i(t),
    \qquad
    h_i(t) = \frac{m_i(t)}{\rho A_i}.
    \label{eq:mi_hi}
\end{equation}
Of course these relations are cyclic (mass depends on height, which depends on mass), so we must choose a fundamental property for our state. We choose the state vector as the tank masses:
\begin{equation}
    x(t) =
    \begin{bmatrix}
        m_1(t)\\ m_2(t)\\ m_3(t)\\ m_4(t)
    \end{bmatrix}.
\end{equation}

\subsection{Flow definitions}

The two pumps produce volumetric flows $F_1(t), F_2(t)$.
Valve split parameters $\gamma_1,\gamma_2\in[0,1]$ distribute the pump flows as
\begin{align}
q_{1,\mathrm{in}}(t) &= \gamma_1 F_1(t), &
q_{4,\mathrm{in}}(t) &= (1-\gamma_1)F_1(t),\\
q_{2,\mathrm{in}}(t) &= \gamma_2 F_2(t), &
q_{3,\mathrm{in}}(t) &= (1-\gamma_2)F_2(t).
\end{align}

Each tank drains through an orifice with area $a_i$.
Using Bernoulli/Torricelli,
\begin{equation}
q_i(t) = a_i \sqrt{2 g h_i(t)},
\qquad i\in\{1,2,3,4\},
\end{equation}
where $g$ is gravitational acceleration.
In the modified 4-tank layout, $q_3$ drains from tank 3 into tank 1, and $q_4$ drains from tank 4 into tank 2.

\subsection{Deterministic dynamics: $\dot x(t)=f(x(t),u(t),d(t),p)$}

We define manipulated variables (MVs) and disturbances as
\begin{equation}
u(t)=\begin{bmatrix}F_1(t)\\F_2(t)\end{bmatrix},
\qquad
d(t)=\begin{bmatrix}F_3(t)\\F_4(t)\end{bmatrix},
\end{equation}
where $F_3,F_4$ are unmeasured inflows entering tanks 3 and 4, respectively.

The parameters are
\begin{equation}
p=
\begin{bmatrix}
    a_i & A_i & \gamma_1 & \gamma_2 & g & \rho
\end{bmatrix}^\top.
\qquad i \in \{1, 2, 3, 4\}
\end{equation}

For a non-reactive system, conservation of mass gives
\begin{equation}
\text{Accumulated} = \text{Influx} - \text{Outflux}.
\end{equation}
For example, for tank 1 over a small interval $\Delta t$,
\begin{equation*}
m_1(t+\Delta t) - m_1(t)
=
\rho q_{1,\mathrm{in}}(t)\Delta t + \rho q_3(t)\Delta t - \rho q_1(t)\Delta t,
\end{equation*}
divide by $\Delta t$ and let $\Delta t\to 0$ to obtain
\begin{equation*}
\frac{dm_1(t)}{dt} = \rho q_{1,\mathrm{in}}(t) + \rho q_3(t) - \rho q_1(t).
\end{equation*}
Applying the same principle to all tanks yields the system mass balances below.


\begin{align*}
\dot m_1(t) &= \rho\big(q_{1,\mathrm{in}}(t) - q_1(t) + q_3(t)\big),\\
\dot m_2(t) &= \rho\big(q_{2,\mathrm{in}}(t) - q_2(t) + q_4(t)\big),\\
\dot m_3(t) &= \rho\big(q_{3,\mathrm{in}}(t) - q_3(t) + F_3(t)\big),\\
\dot m_4(t) &= \rho\big(q_{4,\mathrm{in}}(t) - q_4(t) + F_4(t)\big),
\end{align*}
with $h_i(t)=m_i(t)/(\rho A_i)$ from \eqref{eq:mi_hi}.
This defines $\dot x(t)=f(x(t),u(t),d(t),p)$, which can be written in vector form as:
\begin{equation}\label{eq:system_deterministic}
\dot x(t)=
\rho
\underbrace{
\begin{bmatrix}
-q_1(t)+q_3(t)\\
-q_2(t)+q_4(t)\\
-q_3(t)\\
-q_4(t)
\end{bmatrix}}_\text{Passive dynamics}
+
\rho
\underbrace{
\begin{bmatrix}
q_{\mathrm{in},1}(t)\\
q_{\mathrm{in},2}(t)\\
q_{\mathrm{in},3}(t)\\
q_{\mathrm{in},4}(t)
\end{bmatrix}}_\text{Control input}
+
\rho
\underbrace{
\begin{bmatrix}
0\\
0\\
F_3(t)\\
F_4(t)
\end{bmatrix}}_\text{Disturbance}
\end{equation}

To make the dependence on the inputs more clear, we can extract $x(t)$, $u(t)$, and $d(t)$. Let $k_i = - a_i \sqrt{\frac{2g}{\rho A_i}}$, then:
\begin{equation}\label{eq:system_deterministic}
\dot x(t)=
\rho
\begin{bmatrix}
- k_1 & 0 & k_3 & 0 \\
0 & - k_2 & 0 & k_4 \\
0 & 0 & - k_3 & 0 \\
0 & 0 & 0 & - k_4
\end{bmatrix}
\sqrt{x(t)}
+
\rho
\begin{bmatrix}
\gamma_1 & 0\\
0 & \gamma_2 \\
0 & (1 - \gamma_2) \\
(1 - \gamma_1) & 0
\end{bmatrix}
u(t)
+
\rho
\begin{bmatrix}
0 & 0\\
0 & 0\\
1 & 0\\
0 & 1
\end{bmatrix}
d(t)
\end{equation}



\subsection{Sensor model: $y(t)=g(x(t),p)$}

The sensors measure the tank levels. Using \eqref{eq:mi_hi},
\begin{equation}\label{eq:sensor_deterministic}
y(t)=g(x(t),p)=
\begin{bmatrix}
h_1(t)\\h_2(t)\\h_3(t)\\h_4(t)
\end{bmatrix}
=
\begin{bmatrix}
\frac{m_1(t)}{\rho A_1}\\[4pt]
\frac{m_2(t)}{\rho A_2}\\[4pt]
\frac{m_3(t)}{\rho A_3}\\[4pt]
\frac{m_4(t)}{\rho A_4}
\end{bmatrix}.
\end{equation}

\subsection{Output model: $z(t)=h(x(t),p)$}

In the control problem, the outputs of interest are the lower-tank levels (tanks 1 and 2):
\begin{equation}
z(t)=h(x(t),p)=
\begin{bmatrix}
h_1(t)\\h_2(t)
\end{bmatrix}
=
\begin{bmatrix}
\frac{m_1(t)}{\rho A_1}\\[4pt]
\frac{m_2(t)}{\rho A_2}
\end{bmatrix}.
\end{equation}

\section{Stochastic Nonlinear Model}

In the previous section we derived the deterministic nonlinear model from the system dynamics. Here we keep the same nonlinear plant model, but assume the disturbance inputs are stochastic and \emph{piecewise constant}. Likewise, we introduce measurement uncertainty in the model.

\subsection{Piecewise-constant disturbance model and dynamics}

Let the disturbance be constant on intervals $[t_k,t_{k+1})$:
\begin{equation}
d(t) = d_k, \qquad t_k \le t < t_{k+1},
\qquad
d_k =
\begin{bmatrix}
0\\
0\\
F_{3,k}\\
F_{4,k}
\end{bmatrix}.
\end{equation}
A common choice is to model $\{d_k\}$ as a random walk $d_{k+1}=d_k+w_k$ with $w_k\sim\mathcal{N}(0,Q_d)$ if slow drift is expected. For experimentation purposes the sequence can also be manually pre-defined.

The system is now given as in $\eqref{eq:system_deterministic}$, but with $d(t)=d_k$.

\subsection{Sensor model with measurement noise}

We again assume the sensors to measure tank levels, as in~\eqref{eq:sensor_deterministic}, but now the measurements are corrupted by additive Gaussian noise:
\begin{equation}
y(t) = g(x(t),p) + v(t),
\qquad
v(t)\sim\mathcal{N}\!\big(0,R_{vv}(p)\big).
\end{equation}

A standard parametrization of the measurement-noise covariance is
\begin{equation}
R_{vv}(p)=
\begin{bmatrix}
\sigma_1^2 & 0 & 0 & 0\\
0 & \sigma_2^2 & 0 & 0\\
0 & 0 & \sigma_3^2 & 0\\
0 & 0 & 0 & \sigma_4^2\\
\end{bmatrix},
\end{equation}
for uncorrelated noise.

\section{Stochastic Nonlinear Model (SDE)}

We now drop the assumption that $d(t)$ must be piecewise constant, turning the system dynamics into an SDE:

\begin{equation}
    \mathrm d x(t) = 
    \underbrace{f(x(t), u(t), d(t), p)\mathrm dt}_\text{Undisturbed system} + 
    \underbrace{\sigma(x(t), u(t), d(t), p)\mathrm d\omega(t)}_\text{Stochastic disturbance}
\end{equation}

Sensor and output functions are as before, but are now acting on the stochastically disturbed system.

\section{Simulation}

\subsection{Deterministic Nonlinear Model}
\begin{figure}
    \centering
    \includegraphics[width=\textwidth]{figures/problem2/F1 and F2 constant_None.pdf}
    \caption{
        Hello
    }\label{fig:}
\end{figure}

\begin{figure}
    \centering
    \includegraphics[width=\textwidth]{figures/problem2/F1 varying, F2 constant_None.pdf}
    \caption{
        Hello
    }\label{fig:}
\end{figure}

\begin{figure}
    \centering
    \includegraphics[width=\textwidth]{figures/problem2/F1 constant, F2 varying_None.pdf}
    \caption{
        Hello
    }\label{fig:}
\end{figure}

\subsection{Stochastic Nonlinear Model}

\begin{figure}
    \centering
    \includegraphics[width=\textwidth]{figures/problem2/F1 and F2 constant_Piecewise Constant.pdf}
    \caption{
        Hello
    }\label{fig:}
\end{figure}

\begin{figure}
    \centering
    \includegraphics[width=\textwidth]{figures/problem2/F1 varying, F2 constant_Piecewise constant.pdf}
    \caption{
        Hello
    }\label{fig:}
\end{figure}

\begin{figure}
    \centering
    \includegraphics[width=\textwidth]{figures/problem2/F1 constant, F2 varying_Piecewise constant.pdf}
    \caption{
        Hello
    }\label{fig:}
\end{figure}

\subsection{Stochastic Nonlinear Model (SDE)}

\begin{figure}
    \centering
    \includegraphics[width=\textwidth]{figures/problem2/F1 and F2 constant_Stochastic.pdf}
    \caption{
        Hello
    }\label{fig:}
\end{figure}

\begin{figure}
    \centering
    \includegraphics[width=\textwidth]{figures/problem2/F1 varying, F2 constant_Stochastic.pdf}
    \caption{
        Hello
    }\label{fig:}
\end{figure}

\begin{figure}
    \centering
    \includegraphics[width=\textwidth]{figures/problem2/F1 constant, F2 varying_Stochastic.pdf}
    \caption{
        Hello
    }\label{fig:}
\end{figure}